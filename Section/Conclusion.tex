From the data that has been gathered, it is clear that several methods of decreasing the bandwidth usage (in essence, lowering the bit rate), exists. It can either be by adjusting the resolution of the transmitted video stream, lowering the frame rates or by converting to video to a gray scale.

This has been demonstrated, both by gathering data from a python prototype implementation, and from simply transcoding video files using ffmpeg. Both opportunities gave us the possibility to investigate the outcomes of each resolution etc, and what implications it could have. 

One consideration to take into account, is if the resolution becomes so low that the video might as well not be transmitted at all. A possible solution to avoid lowering the resolution too much, could be to adjust other parameters, such as the frame rate or removing colors, so the video could remove further redundant bit, resulting in a better compression of the video and a smaller file size.

From the ffmpeg test, it was shown that the resolution that would fit inside the requirements of 300Kbps-10Mbps was from around 240x160 to a bit above 1080x720, when using H.264/MPEG-4 at 30 fps. A higher resolution could be used by lowering the frame rate, which is viewed as acceptable at around 10 frames per second. Furthermore, quality can be kept in the current resolution by converting to a gray scale, which removes some information from the frames. The two latter provide decreasing benefits as the resolution is lowered, since they remove data from frames (or the frames themselves), and if each frames holds less data, there is less benefit to removing it.\\

A sensible suggestion to fit the requirements of 300Kbps-10Mbps, is to go from a fully colored high frame rate high resolution (a bit above 1080x720, since that did not reach the upper limit) video stream to a fully colored low frame rate low resolution (at around 300x200 and 6 frames per second) video stream. This setup will be able to transmit video with a bit rate as low as 325 Kbps, and up to the uppewr bound. It is not suggested to convert to a gray scale, since the benefits in the low end is minimal.\\

Further work will go into an android implementation that is currently in development, which should also encode the video before sending it, giving a closer resemblance to what the setup on a drone might look like.
