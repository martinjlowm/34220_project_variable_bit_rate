First off, the source code for the whole project can be found in the online git repository at \url{https://github.com/martinjlowm/34220_project_variable_bit_rate}.\\

\noindent The ability to quickly switch how much data needs to be transmitted has, in general, many useful applications. In this paper the focus is put on variable bit-rates with regards to video, and specifically the MPEG codec.

There are numerous choices when lowering the bit-rate of a video, and several of these methods will be investigated throughout. To name a few possibilities, one can lower the resolution of the video that is to be transmitted. This will remove a lot of information, and thereby lower the size of each frame. Another way is to lower the frame rate, which again removes information by completely stripping out frames (or omitting them in the first place).

Along with some general theory and suggestions on what could be done and how, there will also be some prototype implementations to show-off some of the ideas in practice. Finally some test data will be introduced, which will give a general idea of the real world practicality of the various ideas. The aim is to demonstrate what things are suitable when aiming for video stream that fits inside a requirement a minimum of 300 Kb/s and a maximum of 10 Mb/s.\\

Finally, while the work has in general been done together, the division of labor is represented by a student id in the header of the section the student was responsible for.
