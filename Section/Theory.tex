To allow a drone to cast a stable video stream to a distant server, the loss of an efficient data rate must be countered. Using MPEG codecs, we can reduce the need of the total transmitted bits. Depending on the video feed and its use cases, the compression strength can be chosen and tested to find a proper value which requires a low bit rate while keeping a desired video quality.

In the case of a drone transmitting a feed, this compression must adapt to the quality of the network. A poor wireless network quality reduces how much can be transmitted and the video may experience stuttering frames. The desired video stream is a smooth, stutterless video with images of high quality. 

\subsection{Compressing video}

\subsubsection{Adjustable parameters}
% forklar de forskellige paramtre man kan justere på - framerate, format, farver, kvalitet etc.

\subsubsection{Subjectivity of compression}
% forklar subjektivitens rolle i billedkompression

\subsubsection{Limitations and known issues}
% forklar hvilke begrænsingener og hvilke udfordringer der er ved billedkompression

\subsection{Compressing video using FFmpeg}
% hvordan kan man komprimere video ved hjælp af FFmpeg

\subsection{Protocols for streaming video}

\subsubsection{TCP}
% WS, RMTP, RTSP

\subsubsection{UDP}
% RTP, RMTP




% wireless loss error correction lower bit rate 